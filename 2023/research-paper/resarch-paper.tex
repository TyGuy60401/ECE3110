\documentclass[12pt]{article}

%
%Margin - 1 inch on all sides
%
\usepackage{hyperref}
\usepackage[letterpaper]{geometry}
\usepackage{times}
\geometry{top=1.0in, bottom=1.0in, left=1.0in, right=1.0in}

%
%Doublespacing
%
\usepackage{setspace}
\singlespacing

%
%Rotating tables (e.g. sideways when too long)
%
\usepackage{rotating}


%
%Fancy-header package to modify header/page numbering (insert last name)
%
\usepackage{fancyhdr}
\pagestyle{fancy}
\lhead{} 
\chead{} 
\rhead{Davis \thepage} 
\lfoot{} 
\cfoot{} 
\rfoot{} 
\renewcommand{\headrulewidth}{0pt} 
\renewcommand{\footrulewidth}{0pt} 
\setlength{\headheight}{14.5pt}
%To make sure we actually have header 0.5in away from top edge
%12pt is one-sixth of an inch. Subtract this from 0.5in to get headsep value
\setlength\headsep{0.333in}

%
%Bibtex and Bibliography
%
\usepackage[american]{babel}
\usepackage{csquotes}
\usepackage[style=mla,backend=biber]{biblatex}
\addbibresource{refs.bib}


%
%Begin document
%
\begin{document}
\begin{flushleft}

%%%%First page name, class, etc
Ty Davis\\
Dr. Justin Jackson\\
ECE 3110\\
8 December 2023\\


%%%%Title
\begin{center}
  Alternative Semiconductor Materials to Silicon
\end{center}



%%%%Changes paragraph indentation to 0.5in
\setlength{\parindent}{0.5in}
%%%%Begin body of paper here

The majority of the content of our ECE 3110 class
has considered the use of silicon in semiconductor
devices, but silicon is not the only semiconductor
material available. Through the course of this paper
I will consider the other types of semiconductor
materials that are used commonly, how and why they're
used, and their differences when compared with
silicon. Semiconductor devices have changed the way
our world works, and further developments are constantly
being researched in an attempt to find devices that
can do more for less energy used. The uses of 
semiconductor devices extend far beyond computation
devices such as computer processors. The robust 
utility of semiconductors means that they've found
their way into just about every electronic device 
we see today, and the magic behind a semiconductor
lies in the fact they possess properties of both a
conductor and an insulator. 

The discovery of these electrical properties of semiconductors
can be attributed to many people and began over 150
years ago. The very first instances of discovering
semiconductor properties in materials can even date
back to Faraday's observations regarding electrical
conductivity in relation to temperature. This study
dates back to the year 1833, but it would be a long
time before people began thinking of semiconductors
as such extremely useful devices. Another important
discovery in the early development of semiconductor
thought includes the experiment of Alexandre-Edmond
Becquerel. When he experimented with the electrical
properties of electrolytes, he one day "noted that,
if one of the electrodes was illuminated with sunlight,
the emf generated between the electrodes increased."
\parencite{TudorJenkins_2005}. These first discoveries
of semiconductor properties opened the gateway to many
new ways to experiment with electricity. Doors were
opened and discoveries began to poor in. Other important
discoveries in the timeline of semiconductors include
observations from Willoughby Smith and Heinrich Hertz,
who had similar findings regarding photoconductivity
in the late 1800s, and another late 1800s discovery
regarding rectification in the contacts between metals
and some oxides and sulfides. Discoveries like these
poured out over the course of several decades and led
to increased thought and innovation in the area.

Moving forward to the 1920s and '30s, there was a significant
increase in demand for radar and other forms of communication
due to the impending war. Under this pressure, and
with the recent discovery of bipolar conduction in
semiconductors (being the flow of current due to both
electrons and holes moving), the scientific scene was
ready for heavy increase in the development of semiconductor
devices. The two primary material subjects of semiconductor
research were silicon and germanium, though many materials
possess the desired qualities in semiconductor behavior.
Discoveries surrounding the p-n junction and its manufacture
paved the way to new applications of diodes that could
support more current flow. This eventually led to the
discovery of transistors. In 1950, Gordon Teal and
his team grew a single crystal of germanium that contained
regions of both p-type and n-type materials. After
some more processing, "[this] crystal was cut into
n-p-n rods and contacts applied to the three regions
of doping. The electrical properties of the transistor
thus made were largely consistent with Shockley's proposed
junction transistor. This transistor was more reliable
than the point contact version, generated less noise
and could handle higher powers" \parencite{TudorJenkins_2005}.

The advent of transistors led the way to many new technologies,
and silicon quickly became the favorite material for
semiconductor development. The main reasons that silicon
is used in integrated circuits today are plenty. Silicon
is one of the most abundant resources found here on
Earth, meaning that it is relatively cheap and was
more readily available for experimentation and testing.
More important than cost, though, was the material
properties and electrical properties that made it so
special. Silicon boasts amazing reliability in diverse
conditions, and has shown the ability to operate as
intended in extreme weather and temperature conditions.
As silicon was adopted in the industry, manufacturers
would decide to make devices out of silicon so that
they could be compatible with already built devices,
and so that the manufacturing processes could source
from others as well.

The current methods for manufacturing silicon chips
have been through many renditions and revisions to get
where they are today. Each manufacturing company may have
their own processes to build the chips, but the process
roughly follows this same outline. Silicon chips are
cut from wafers of silicon. After a purification process
to find produce a raw crystal of silicon, thin wafers
are sliced from a large ingot. They are then cleansed
and polished to make sure that the surface is free
and defect free. Any small defects in the wafer now
can prevent operation in the integrated chip once it's
done. Once they've produced a clean silicon wafer it
undergoes a process that produces an oxidized SiO$_2$
layer on top of the silicon which acts as an insulating
material. The insulated silicon wafer then undergoes
a photolithography stage which defines the patterns
for the various components on the integrated circuit.
After an etching process, the silicon wafer undergoes
what is possibly the most important step when considering
the use of semiconductors. This is the doping stage, where
impurities are introduced into the silicon and modify its
electrical properties. Creating separate regions of p-type
and n-type concentrated areas allows the semiconductor
silicon to act as diodes and transistors. These produce
the computational logic of the chip and allow it to
work as intended. Often times, a silicon wafer undergoes
lithography, etching, doping and deposition multiple
times to correctly build the proper design on the 
silicon chip.

These manufacturing techniques have been developed
over years of experimentation with silicon chips, thus,
silicon quickly became king of the semiconductor field.
The discoveries, developments, and innovations in the
realm of electronics have increased greatly in this
era of silicon dominated semiconductor materials, but,
as we saw from years of electronic research and experimentation,
silicon isn't alone in its electrical properties. Other
materials are commonly used as semiconductors in electronics
today, including germanium, gallium arsenide, gallium
nitride, and others. Their strengths are separate from
those of silicon, thus they find their own place in
the world of semiconductor use. Many semiconductors,
including the ones just previously listed, are made
of alloys consisted of Group III and Group V elements
on the periodic table. Gallium belongs to Group III
and has been paired in experimentation with several
Group V elements and found to have unique properties
that make it especially useful in specific areas, such
as the area of optoelectronics.

Gallium arsenide (GaAs) has been found very useful
in the area of optoelectronics. It has a higher electron
mobility than silicon, which makes it useful in high
frequency applications, such as radar systems, satellite
communication devices, etc. Research is currently being
done in the realm of building computer processors out
of GaAs because of its strengths in these areas which
could allow it to overcome some shortcomings of silicon-based
semiconductor devices. The major roadblocks that keep
silicon in the spotlight when it comes to semiconductor
devices, is the cost of materials and manufacturing.
One of the most common processes required for the manufacturing
of GaAs semiconductor devices is the use of molecular
beam epitaxy.

Molecular beam epitaxy (MBE) isn't just a process used
for the creation of GaAs semiconductor devices, it
can also be used for other Group III-V semiconductor
materials. The MBE process must be completed in an
ultra-high vacuum (UHV) environment, which is one of
the leading factors in how expensive the process is.
However, without a UHV environment, the process would
allow impurities into the semiconductor material, and
it would result in a poor outcome, so it is essential.
Similar to silicon wafers, after a crystal structure
is made from the semiconductor material, thin wafers
are cut from the crystal and prepared for processing.
Once the wafers of material are prepared, the effusion
cells are heated, evaporating the materials that will
be deposited on the surface of the wafer. This forms
molecular beams which are then directed toward the
wafer in a very controlled manner, and eventually growth
of the material is developed on the substrate. "[An
evaporated material] is deposited by Atomic Layer Deposition
(ALD) process for advantages like high quality and
purity" \parencite{MBE}. While MBE is very expensive,
one of the advantages is the precision in the manufacturing
process. Well-designed MBE processes can have enough
control to allow the process to be specific to each
atomic layer of growth on the substrate. Often, these
steps are repeated multiple times to ensure that the
process is done well and completely. As previously
stated, one of the contributing factors to the increased
cost in the MBE manufacturing process is the necessity
of an ultra-high vacuum environment. Another contributing
factor to the elevated cost is the amount of time it
takes to grow the layers. It is often a timely process,
and that doesn't make it cheap.

Each development in the semiconductor world has its
own strengths. Another realm of semiconductors being
researched now is the use of carbon nanotubes. These
small nanotubes, constructed almost entirely out of
carbon, are used to make flexible electronic circuit
devices, and have shown excellent properties when it
comes to conductivity. By changing the way that the
tubes are shaped and aligned, one can moderate the
electrical properties of the nanotubes. This means
that they can be used in electronic devices as well
as semiconductor devices. Some of the defining features
of carbon nanotubes that make them desirable to use
are their absolutely tiny size. As Moore's law comes
to a close in recent years, it's getting more difficult
to build silicon chips with even smaller transistors,
and carbon nanotubes can allow for even smaller transistor
networks and circuits. Another interesting thing about
carbon nanotubes is their exceptional mechanical strength.
Robust electronic devices that are built to withstand
flexible designs are possible with carbon nanotube
circuits. Being made of carbon means that it resists chemical
changes as well, and it is more resilient to environmental
factors that could damage semiconductors made from other 
materials.

Just as with other alternatives to silicon in the semiconductor
world, these benefits also come with costs. The materials
for building carbon nanotubes, including the carbon
resources as well as catalysts that aid in the manufacturing
process, are more scarce and thus more expensive. The
lack of development in the space also means that manufacturing
processes are more expensive. These costs result in
fewer resources available to create devices with carbon
nanotubes, and thus it lacks the economies of scale
that silicon devices benefit from. The most common
practice in manufacturing carbon nanotube devices is
known as chemical vapor deposition.

In chemical vapor deposition (CVD), a process which
is used in the manufacturing of several types of semiconductor
devices, a thin film or coating is put onto a substrate
by evaporating several materials, including methane,
ethylene, and some hydrocarbons, in a chamber and then
allowing those gases to collect on a substrate in thin
layers. Once the layers are collected on the substrate,
they undergo chemical reactions that produce the desired
film on the substrate. All of these processes result
in some undesirable byproducts that need to be removed
at the end of the process. By specifically controlling
each element of the CVD process, carbon nanotubes can
be created and designed to very specific parameters.
This amount of control is expensive, but allows for the
specific design and application of carbon nanotubes in 
different use cases, including semiconductor devices,
especially those which are flexible.

Another interesting branch of semiconductor materials
are those labelled as \emph{organic} semiconductors.
You may have heard of the newer TV display technologies
featuring \emph{organic} LEDs (or OLEDs). These are
devices made from organic compounds that allow for
unique semiconductor properties. OLEDs have desirable
qualities in their own right. While organic semiconductors
don't have the best properties when it comes to processing
power and other high-speed or high-frequency use cases,
they can be found in consumer electronics because of
their use in displays. OLEDs show very dynamic colors
and allow displays to have more contrast as well. The
reason that they don't showcase lots of speed is because
the electron mobility is often reduced in organic semiconductor
materials. Organic semiconductors aren't only used
in OLEDs, they have some other use cases as well. 
Some are used in organic solar cells, and photodetectors.
In this way they differ from silicon because silicon
has poor optoelectronic qualities.

A large realm of electronics that heavily utilizes
semiconductors is the realm of optoelectronics. There
are many facets of optoelectronics that make it such
a widely influential area in our world. Optoelectronics
present a way that we are able to directly interact
with electronic circuits. As humans, we receive the
majority of the information we process through our
eyes, so optoelectronics has always been a large priority
for the development of any industry, including that
of electronics. The first optoelectronic devices to
be developed were light emitting diodes (LEDs) which
were made with silicon carbide point contact diodes
in 1923 by Oleg Losev. The LEDs were weak, as were
the solar cells of the time, and progress was slow
in the development and discovery within the field,
but it moved on. Discoveries continued within the field
for several decades, and "[the] realization of the
wide-gap window effect was very important for photodetectors,
solar cells, and LED applications. It permitted us
to broaden considerably and to control precisely the
spectral region in solar cells and photodetectors and
to improve drastically the efficiency for LEDs" \parencite{ZhoresOpto}.
These significant differences in the train of thought
led to many more innovations in the realm of optoelectronics.

Silicon isn't often used in optoelectronic devices
because of its poor ability to produce or receive light.
This is because of its indirect bandgap. The amount
of energy required for an electron to move from the
valence band to the conduction band isn't correlated
with the energy of a photon that it can absorb or emit.
Because it has limitations in this aspect, other materials
are often used in its stead for optoelectronic devices.
Some of these other semiconductor materials include
germanium, GaAs, indium phosphide (InP), other III-V
semiconductor materials, and even some II-VI semiconductors
such as cadmium selenide. When it comes to optoelectronic
devices and their requirements, there often isn't a
clear choice because each material has specific properties
that will react differently in these conditions. Different
materials can result in different wavelengths of light
absorbed or emitted. This can mean that separate materials,
when used to produce LEDs, will make different colors
of light. It was a surprise to me to find that the
white LED was discovered in the late '90s, when 
experimentation with LEDs dates all the way back to the
'20s.

One of the most currently demanding areas of optoelectronic
devices is the creation and use of solar cells used
in solar panels. Photovoltaic cells allow us to transfer
the energy of light into electricity and power other
electronic devices that require energy. Studies and
manufacturing within this field have shown that converting
from sunlight to energy is not a trivial task. Despite
silicon's indirect bandgap and limited efficiency in
the optoelectronic world, it is still the most widely
used material in optoelectronic devices. This is largely
because of the large infrastructure built around silicon
devices, and because it is often reliable in extreme
conditions. Reliability is certainly important as solar
cells placed indoors are worthless in energy production.
This is often where other photovoltaic cells are lacking,
such as Perovskite solar cells. These solar cells are
made with a material that has been shown to rival silicon
in efficiency, and boast lower manufacturing costs,
but the relative stability in real world cases makes
it very difficult to justify in real world instances.
Perovskite solar cells are unique because they are
made from perovskite crystal structures such as methylammonium
lead iodide, (CH$_3$NH$_3$PbI$_3$). Other concerns
with these materials include their toxicity, which
would especially prove to be a problem if they aren't
reliable and would have to be disposed of often.

Because of its abundance and manufacturing infrastructure,
silicon remains the first choice for many semiconductor
applications today. We are tasked with whether silicon
will remain the future of semiconductor devices. Because
of its infrastructure and current development in the
field, it will always play a role in the world of electronics,
but in some areas it is limited. It has relatively
low electron mobility when compared with some other
semiconductor materials, and is rather rigid, making
it difficult to use in flexible devices. It has an
indirect bandgap which limits its use in optoelectronic
devices such as LEDs and lasers. Despite these limitations,
there is no clear replacement for silicon in the current
electronic landscape. Situations may arise that demand
more from our semiconductor devices, and in those situations
there may be discoveries and innovations that lead
to materials that can outperform silicon in each task,
but silicon will likely remain the favorite material
for so many applications.



\newpage
\nocite{*}
\printbibliography

\end{flushleft}
\end{document}