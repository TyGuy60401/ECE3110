\documentclass{article}
\usepackage{../report}
\graphicspath{./imgs}
\title{Lab 7 Report}
\date{November 6, 2023}

\begin{document}
\maketitle
\section{Introduction}

In this lab our task is to study the performance of
an NMOS "Source Follower" or common-drain amplifier.
We will first perform an analysis of the DC component 
of the circuit, then we will analyze its performance
with small signals. Our goal is to define an amplifier
with a gain of 0.8 \unit{\volt/\volt}, using voltage supplies
of V$_+=-$V$_{-}=15$ \si\volt, R$_{\text{sig}}=50$ \si\ohm, and
R$_{\text{G}}=10$ \si\kohm. The main circuit to be analyzed
is shown in Figure \ref{fig:main-circuit}.

\begin{figure}[!ht]
  \begin{center}
  \begin{circuitikz}[american]
    \ctikzset{tripoles/mos style/arrows}
    \def\killdepth#1{{\raisebox{0pt}[\height][0pt]{#1}}}

    \draw (2,2) node[nmos](Q1){\killdepth{N$_1$}};
    \draw (0,0) node[ground]{}
    to[resistor, l=10 \si\kohm] (0,2) to[short, *-] (Q1.G);

    \draw (0,2) to[capacitor, l_=C$_{\text{C1}}$] (-2,2)
    to[resistor, l_=R$_{\text{sig}}$] (-2,0.5)
    to[voltage source, l_=V$_{\text{sig}}$] (-2,-1)
    node[ground]{};

    \draw (Q1.D) -- (2,3) node[vcc](Vcc){V$_{cc}=15$ V};
    \draw (Q1.S) to[short, -*] (2,1)
    to[resistor, l=R$_S$] (2,-1) node[vee](Vee){V$_{ee}=-15$ V};
    
    \draw (2,1) to[capacitor, l=C$_{\text{C2}}$] (4,1)
    to[resistor, l=R$_L$] (4,-1) node[ground]{};

    \draw (4,1) to[short, *-o] (5,1);
    \draw (5,1) to[open, v=$V_o$] (5,-1.5);
  \end{circuitikz}
  \caption{Main Circuit for Analysis in this Lab}
  \label{fig:main-circuit}
  \end{center}
\end{figure}

\section{DC Analysis}

\begin{figure}[!ht]
  \begin{center}
  \begin{circuitikz}[american]
    \ctikzset{tripoles/mos style/arrows}
    \def\killdepth#1{{\raisebox{0pt}[\height][0pt]{#1}}}

    \draw (2,2) node[nmos](Q1){\killdepth{N$_1$}};
    \draw (0,0) node[ground]{}
    to[resistor, l=10 \si\kohm] (0,2) -- (Q1.G);

    \draw (Q1.D) -- (2,3) node[vcc](Vcc){V$_{cc}=15$ V};
    \draw (Q1.S) -- (2,1) node[right=3mm] (Vs){V$_{S}$};
    \draw (2,1) to[resistor, l=R$_S$] (2,-1) node[vee](Vee){V$_{ee}=-15$ V};
  \end{circuitikz}
  \caption{Circuit Reduced for Small Signal Analysis}
  \label{fig:DC-Analysis}
  \end{center}
\end{figure}

Refer to the circuit in Figure \ref{fig:DC-Analysis}
to see a circuit that has reduced for DC Analysis by
removing the components that occur outside the capacitors.
From the last lab we know that the k$_n$ value is k$_n=0.23781$
and the threshold voltage is V$_T=2.1$ \si\V. We are
designing the circuit such that I$_D=1$ \si\mA, and
using the equation $I_D=\frac12k_n(V_{OV})^2$ we can
solve for $V_{OV}$. For $V_{OV}$ we get the value $V_{OV}=0.0917$
\si\V, and that results in a $V_{GS}$ value of $V_{GS}=2.1917$
\si\V. Because there is no current through the gate,
the current through R$_G$ is 0 \si\A and the voltage
at $V_G=0$ \si\V. Using this we can find that the voltage 
at $V_S$ is $V_S=-2.1917$ \si\V, and the resistor we
should use to obtain that value and the appropriate 
current is R$_S=12.808$ \si\kohm.

\section{Small Signal Analysis}

\begin{figure}[!ht]
  \begin{center}
  \begin{circuitikz}[american]
    \ctikzset{tripoles/mos style/arrows}
    \def\killdepth#1{{\raisebox{0pt}[\height][0pt]{#1}}}

    \draw (0,0) node[ground]{}
    to[resistor, l=10 \si\kohm] (0,2)-- (-1.5,2)
    to[resistor, l_=R$_{\text{sig}}$] (-1.5,0.5)
    to[voltage source, l_=V$_{\text{sig}}$] (-1.5,-1)
    node[ground]{};

    \draw (1,2) node[above=2mm] {V$_i$};
    \draw (0,2) to[short, *-*] (1,2) -- (2,2) --
    (2,1) to[short, -*] (3,1)
    to[resistor, l=1/gm] (3, -1)
    -- (3, -2) -- (2, -2) to[resistor, l_=12.808 \si\kohm] (2, -4)
    to[short, -*] (3,-4) -- (4, -4) to[resistor, l_=R$_L$] (4, -2)
    to[short, -*] (3, -2);
    \draw (3,-4) -- (3, -5) node[ground]{};

    \draw (5,3) node[ground]{} -- (3,3)
    to[cI, l=gmV$_{GS}$] (3,1);

    \draw (3,-1) to[short, *-o] (2,-1);
    \draw (2,1) to[open, v=V$_{GS}$] (2,-1);
    


  \end{circuitikz}
  \caption{Circuit for AC Analysis}
  \label{fig:AC-Analysis}
  \end{center}
\end{figure}
\end{document}
