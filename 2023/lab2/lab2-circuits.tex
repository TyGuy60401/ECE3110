\documentclass{article}
\usepackage{geometry}
\usepackage{circuitikz}
\usepackage{siunitx}

\geometry{margin=1in}

\begin{document}

\section{Circuits in Latex}

% Circuit 1
\begin{center}
  \begin{figure}[h!]
  \centering
  \begin{minipage}{.45\textwidth}
    \label{circuit1}
    \begin{circuitikz}[american] \draw
    (0,0) to[sinusoidal voltage source, v^<=$V$] ++(0,4)
      to[full diode, l=1N4001] (4,4) 
      to[R, l_= 10 \si{\kohm}]
      (4,0) -- (0,0)
    (0,0) node[ground]{}
    (4,4) to[short, *-o] (5,4) 
    (4,0) to[short, *-o] (5,0) 
    (5,4) to[open, v=$V_0$] (5,0)
    ;
    \end{circuitikz}
    \caption{Half-wave Rectifier}
  \end{minipage}
  %
  % Circuit 2
  \begin{minipage}{.5\textwidth}
    \begin{circuitikz}[american]
    \draw (0,0) to[sV, v^<=$V$] (0,4)
      to[full diode, l=1N4001] (4,4) node[label={above:V1}]{}
      to[capacitor, l_=1.5 \si{\uF}, *-*] (4,0)
      -- (0,0)
    (0, 0) node[ground]{}
    ;
    \draw (4,4) -- (6,4)
      to[resistor, l_=10 \si{\kohm}] (6,0)
      -- (4,0)
    ;
    \draw
    (6,4) to[short, *-o] (7,4)
    (6,0) to[short, *-o] (7,0)
    (7,4) to[open, v=$V_0$] (7,0)
    ;
    \end{circuitikz}
    \caption{Peak Rectifier Circuit}
    \label{circuit2}
  \end{minipage}
  %
  % Circuit 3
  \begin{minipage}{.45\textwidth}
    \vspace{20pt}
    \begin{circuitikz}[american]
    \draw node[ground]{}
      (0,0) to[sV, v^<=$V$] (0,2)
      -- (2,2) node[op amp, noinv input up, anchor=+](opamp){\texttt{741}}
      (opamp.out) node[right]{} -- (5,1.5)
      to[full diode, l=1N4001, -*] (5,0)
      (opamp.-) node[left]{} -- (2,0)
      -- (5,0) to[resistor, l_=10 \si{\kohm}] (5, -2)
      node[ground]{}
    (5,0) to[short, -o] (6,0)
    (6,0) to[open, v=$V_0$] (6,-2.5)
    ;
    \end{circuitikz}
    \caption{Precision Rectifier}
    \label{circuit3}
  \end{minipage}
  \end{figure}
\end{center}


\end{document}