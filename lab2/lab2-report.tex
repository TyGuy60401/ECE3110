\documentclass{article}
\usepackage{geometry}
\usepackage{graphicx}
\graphicspath{ {./imgs/} }
\usepackage{circuitikz}
\usepackage{siunitx}

\geometry{margin=1in}
\author{Ty Davis}
\title{Lab 2 Report}
\newcommand{\twopics}[4]{
\begin{figure}[h!]
\begin{center}
  \begin{minipage}{.45\textwidth}
    \includegraphics[width=\textwidth]{#1}
  \end{minipage}
  \begin{minipage}{.45\textwidth}
    \includegraphics[width=\textwidth]{#2}
  \end{minipage}
  \caption{#4}
  \label{#3}
\end{center}
\end{figure}
}

\begin{document}
\maketitle

\section{Introduction}
In this lab we will be simulating and measuring voltages
of various rectifier circuits featuring diodes. We'll be 
learning about the purposes of rectifier circuits, and 
we'll be analyzing the features of their output. The three
circuits that we are analyzing are shown in Figures \ref{fig:circuit1},
\ref{fig:circuit2}, and \ref{fig:circuit3}. We will simulate and 
analyze each circuit under two conditions to better 
understand the functions of each circuit. 
% \section{Circuits in Latex}

% Circuit 1
\begin{center}
  \begin{figure}[h!]
  \centering
  \begin{minipage}{.45\textwidth}
    \begin{circuitikz}[american] \draw
    (0,0) to[sinusoidal voltage source, v^<=$V$] ++(0,4)
      to[full diode, l=1N4001] (4,4) 
      % to[R, l_= 10 \unit{\kilo\ohm}]
      to[R, l_= 10 \si{\kohm}]
      (4,0) -- (0,0)
    (0,0) node[ground]{}
    (4,4) to[short, *-o] (5,4) 
    (4,0) to[short, *-o] (5,0) 
    (5,4) to[open, v=$V_0$] (5,0)
    ;
    \end{circuitikz}
    \caption{Half-wave Rectifier}
    \label{fig:circuit1}
  \end{minipage}
  %
  % Circuit 2
  \begin{minipage}{.5\textwidth}
    \begin{circuitikz}[american]
    \draw (0,0) to[sV, v^<=$V$] (0,4)
      to[full diode, l=1N4001] (4,4) node[label={above:V1}]{}
      to[capacitor, l_=1.5 \si{\uF}, *-*] (4,0)
      -- (0,0)
    (0, 0) node[ground]{}
    ;
    \draw (4,4) -- (6,4)
      to[resistor, l_=47 \si{\kohm}] (6,0)
      -- (4,0)
    ;
    \draw
    (6,4) to[short, *-o] (7,4)
    (6,0) to[short, *-o] (7,0)
    (7,4) to[open, v=$V_0$] (7,0)
    ;
    \end{circuitikz}
    \caption{Peak Rectifier Circuit}
    \label{fig:circuit2}
  \end{minipage}
  %
  % Circuit 3
  \begin{minipage}{.45\textwidth}
    \vspace{20pt}
    \begin{circuitikz}[american]
    \draw node[ground]{}
      (0,0) to[sV, v^<=$V$] (0,2)
      -- (2,2) node[op amp, noinv input up, anchor=+](opamp){\texttt{741}}
      (opamp.out) node[right]{} -- (5,1.5)
      to[full diode, l=1N4001, -*] (5,0)
      (opamp.-) node[left]{} -- (2,0)
      -- (5,0) to[resistor, l_=10 \si{\kohm}] (5, -2)
      node[ground]{}
    (5,0) to[short, -o] (6,0)
    (6,0) to[open, v=$V_0$] (6,-2.5)
    ;
    \end{circuitikz}
    \caption{Precision Rectifier}
    \label{fig:circuit3}
  \end{minipage}
  \end{figure}
\end{center}

In this report there will be a separate section for each
type of rectifier in which we'll analyze the built circuit
and measurements taken, as well as the computer analysis done
in Multisim. 

\section{Half-wave Rectifier} The first circuit, shown
by Figure \ref{fig:circuit1}, is a half-wave rectifier.
The diode is preventing current from flowing towards
the voltage supply, and therefore will only allow a
positive voltage over $V_0$. The result is an output
waveform that returns only the positive voltage from
the input. With an input sinusoidal voltage of 10~V$_{pk-pk}$,
we measure an output half-sine wave that is decreased
just slightly because of the operating voltage of the
diode. In Figure \ref{fig:output1} you can see that
the peak voltage of the output signal is 9.4 V when
10 V input was used, a decrease of about 0.6 V. This
follows closely with the analysis provided by the constant
drop model, which says that a drop of about 0.7 V occurs
across a forward biased diode. The output voltage minimum
sits at 0 V.

In Figure \ref{fig:output1} we can see that the half-wave
rectifier showed very poor performance for a low input
sinusoidal signal. The output was just a small bump, and
only rose above 0 V for a small portion of the positive
range of input voltage. The minimum was 0 V, and the maximum
was 113 mV.

Compare these results with those from the simulation
as captured in Figure \ref{fig:sim1}.

\twopics{scope/Circuit1-10V-max2}{scope/Circuit1-0.5V-max2}{fig:output1}
{Half-wave Rectifier Results - Left: 10~V$_{pk}$ Input - Right: 0.5~V$_{pk}$ Input}

\twopics{multisim/circuit1-10V}{multisim/circuit1-10V}{fig:sim1}
{Half-wave Rectifier Simulation - Left: 10~V$_{pk}$ Input - Right: 0.5~V$_{pk}$ Input}

\section{Peak Rectifier}

Looking at Figure \ref{fig:output2} we can see the
performance of the peak rectifier with two different
resistor values being placed in parallel with the capacitor.
The capacitor allows the voltage to remain high when
the sinusoidal input voltage dips below its peak. When
a higher resistance was placed in parallel with the
capacitor the high output signal was maintained much
better than with a lower value resistor. In both cases
an input signal of 10~V$_{pk}$ was used. You can see
that the peak-to-peak voltage when the 47~\si{\kohm}
resistor was used was about half the peak-to-peak range
compared to when the 4.7~k$\Omega$ resistor was used.

Compare our measured results with the simulated ones
as found in Figure \ref{fig:sim2}.

\twopics{scope/Circuit2-47k-ripple-max}{scope/Circuit2-4.7k-max-ripple}{fig:output2}
{Peak Rectifier Results - Left: 47 k$\Omega$ Resistor - Right: 4.7~k$\Omega$ Resistor}

\twopics{multisim/circuit2-47k}{multisim/circuit2-4.7k}{fig:sim2}
{Peak Rectifier Simulation - Left: 47 k$\Omega$ Resistor - Right: 4.7~k$\Omega$ Resistor}

\section{Precision Rectifier}

The precision rectifier shown in Figure \ref{fig:circuit3} 
utilizes an operational amplifier to accomplish the same
task as the circuit in Figure \ref{fig:circuit1}, but
it does the jump better. The configuration ensures that
the output voltage stays as close to the input voltage
when the input is positive, and it remains at 0 V when
the input is negative. You can see from the results in 
Figure \ref{fig:output3} and the simulation in Figure
\ref{fig:sim3} that the precision rectifier does a
much better job at keeping the output voltage
much closer to the value of the input voltage when it is
being passed through. Here there is no 0.7~V drop
that usually occurs with the use of the diode, even at
a low input signal of just 500~mV.

\twopics{scope/Circuit3-10V-max1-max2}{scope/Circuit3-0.5V-max1-max2}{fig:output3}
{Precision Rectifier Results - Left: 10~V$_{pk}$ - Right: 0.5~v$_{pk}$}

\twopics{multisim/circuit3-10V}{multisim/circuit3-0.5V}{fig:sim3}
{Precision Rectifier Simulation - Left: 10~V$_{pk}$ - Right: 0.5~v$_{pk}$}

\section{Post-measurement Exercise}

See the table in Figure \ref{table:resistors} for our
resistor values.

\begin{figure}[h!]
\begin{center}
  \begin{tabular}{c c c}
    \textbf{Circuit} & \textbf{Target Value} & \textbf{Measured Value}\\
    1 & 10 \si\kohm & 10.018 \si\kohm\\
    2 & 47 \si\kohm & 46.24 \si\kohm\\
    2 & 4.7 \si\kohm & 4.604 \si\kohm\\
    3 & 10 \si\kohm & 10.018 \si\kohm
  \end{tabular}
\end{center}
\caption{Measured Resistor Values}
\label{table:resistors}
\end{figure}

Images of the simulated circuits are omitted
for brevity, and because they appear the same
as the simulations that were done with precise
resistor values.

\section{Summary}

One of the most important uses of semiconductor devices
is the process of rectification. This allows us to
take an alternating voltage or current input and translate
it to direct voltage or current with little loss in
energy. Simple rectifiers such as the one in Figures
\ref{fig:circuit1} and \ref{fig:circuit2} illustrate
that decent results can be achieved in the right conditions
with very few components. However, a simple operational
amplifier can take things to the next level.

\end{document}