\documentclass{article}
\usepackage{geometry}
\usepackage{graphicx}
\usepackage{circuitikz}
\usepackage{siunitx}

\geometry{margin=1in}
\author{Ty Davis}
\title{Lab 2 Report}

\begin{document}
\maketitle

\section{Introduction}
In this lab we will be simulating and measuring voltages
of various rectifier circuits featuring diodes. We'll be 
learning about the purposes of rectifier circuits, and 
we'll be analyzing the features of their output. The three
circuits that we are analyzing are shown in Figures \ref{circuit1},
\ref{circuit2}, and \ref{circuit3}. We will simulate and 
analyze each circuit under two conditions to better 
understand the functions of each circuit. 
% \section{Circuits in Latex}

% Circuit 1
\begin{center}
  \begin{figure}[h!]
  \centering
  \begin{minipage}{.45\textwidth}
    \label{circuit1}
    \begin{circuitikz}[american] \draw
    (0,0) to[sinusoidal voltage source, v^<=$V$] ++(0,4)
      to[full diode, l=1N4001] (4,4) 
      % to[R, l_= 10 \unit{\kilo\ohm}]
      to[R, l_= 10 \si{\kohm}]
      (4,0) -- (0,0)
    (0,0) node[ground]{}
    (4,4) to[short, *-o] (5,4) 
    (4,0) to[short, *-o] (5,0) 
    (5,4) to[open, v=$V_0$] (5,0)
    ;
    \end{circuitikz}
    \caption{Half-wave Rectifier}
  \end{minipage}
  %
  % Circuit 2
  \begin{minipage}{.5\textwidth}
    \begin{circuitikz}[american]
    \draw (0,0) to[sV, v^<=$V$] (0,4)
      to[full diode, l=1N4001] (4,4) node[label={above:V1}]{}
      to[capacitor, l_=1.5 \si{\uF}, *-*] (4,0)
      -- (0,0)
    (0, 0) node[ground]{}
    ;
    \draw (4,4) -- (6,4)
      to[resistor, l_=10 \si{\kohm}] (6,0)
      -- (4,0)
    ;
    \draw
    (6,4) to[short, *-o] (7,4)
    (6,0) to[short, *-o] (7,0)
    (7,4) to[open, v=$V_0$] (7,0)
    ;
    \end{circuitikz}
    \caption{Peak Rectifier Circuit}
    \label{circuit2}
  \end{minipage}
  %
  % Circuit 3
  \begin{minipage}{.45\textwidth}
    \vspace{20pt}
    \begin{circuitikz}[american]
    \draw node[ground]{}
      (0,0) to[sV, v^<=$V$] (0,2)
      -- (2,2) node[op amp, noinv input up, anchor=+](opamp){\texttt{741}}
      (opamp.out) node[right]{} -- (5,1.5)
      to[full diode, l=1N4001, -*] (5,0)
      (opamp.-) node[left]{} -- (2,0)
      -- (5,0) to[resistor, l_=10 \si{\kohm}] (5, -2)
      node[ground]{}
    (5,0) to[short, -o] (6,0)
    (6,0) to[open, v=$V_0$] (6,-2.5)
    ;
    \end{circuitikz}
    \caption{Precision Rectifier}
    \label{circuit3}
  \end{minipage}
  \end{figure}
\end{center}

In this report there will be a separate section for each
type of rectifier in which we'll analyze the built circuit
and measurements taken, as well as the computer analysis done
in Multisim. 

\section{Half-wave Rectifier}
The first circuit, shown by Figure \ref{circuit1}, is 
a half-wave rectifier. The diode is preventing current
from flowing towards the voltage supply, and therefore
will only allow a positive voltage over $V_0$. The result
is an output waveform that returns only the positive
voltage from the input. With an input sinusoidal voltage 
of 10 V$_{pk-pk}$, we measure an output half-sine wave
that is decreased just slightly because of the operating
voltage of the diode. See Figure \ref{output1}.

\section{Summary}

\end{document}