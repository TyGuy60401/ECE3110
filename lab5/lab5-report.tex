\documentclass{article}
\usepackage{../report}
\graphicspath{ {./imgs} }
\title{Lab 5 Report}

\begin{document}
\maketitle
\section{Introduction}
Through paper calculation, computer simulation, and physical
measurements we will analyze the DC behavior of three similar 
circuits featuring an NMOS transistor. The first is an NMOS
biased in the saturation region, as well as an NMOS biased in 
the triode region, and finally a diode connected NMOS
transistor. 


\begin{figure}[!h]
  \begin{center}
  \begin{circuitikz}[american] 
    \ctikzset{tripoles/mos style/arrows}
    \def\killdepth#1{{\raisebox{0pt}[\height][0pt]{#1}}}

    \draw (0,0) node[ground]{}
    to[voltage source, v=$15$ V] (0,2)
    to[resistor, l=R$_2$] (0,3.2)
    -- (0,5) node[label={left:V$_G$}]{}
    -- (0,6.8)
    to[resistor, l=R$_1$] (0,8)
    to[voltage source, v=$15$ V] (0,10)
     -- (-2,10) -- (-2, 9) node[ground]{};

    \draw (0,5) -- (1.05,5);
    \draw (2,5) node[nmos](Q1){\killdepth{N$_1$}};
    \draw (2,5.75) node[label={right:V$_D$}]{} --
    (2,6.8) to[resistor, l_=R$_D$] (2,8)
    to[voltage source, v_=$15$ V] (2,10)
    -- (4,10) -- (4,9) node[ground]{};
    
    \draw (2,0) node[ground]{}
    to[voltage source, v_=$15$ V] (2,2)
    to[resistor, l_=R$_S$] (2,3.2)
    -- (2, 4.25) node[label={right:V$_S$}]{};

    % The open voltage labels
    \draw (3.5,6.5) to[open, v_=V$_{DS}$] (3.5,3.5);
    \draw (1,4.7) to[open, v=V$_{GS}$] (1.7,3.8);
    \draw (1.7,6.2) to [open, v=V$_{DG}$] (1,5.3);

    
  
  \end{circuitikz}
  \caption{The Circuit we Analyze in this Lab}
  \label{fig:maincircuit}
  \end{center}
\end{figure}
The transistor labeled N$_1$ is the same transistor as 
we used in the last lab, of the type 2N7000. For the 
first circuit we were tasked with designing the circuit
such that I$_D=1$ mA, V$_G=0$ V, and V$_D=+5$ V. We need
to select the resistor values that will satisfy such a 
circuit. The rail and drain voltages that we used were
+15~V and -15~V.

\section{Saturation Region Analysis/Experiment}
\subsection{Hand Calculations}
The calculations are made for the circuit shown in 
Figure \ref{fig:maincircuit}.

While searching for a data sheet of the 2N7000G on 
the internet we found this source on
\href{https://ww1.microchip.com/downloads/en/devicedoc/2n7000-n-channel-enhancement-mode-vertical-dmos-fet-data-sheet-20005695a.pdf}{Microchip's website}.
From this data sheet we were able to find that the threshold
voltage of the 2N7000 transistor is around 2.1 V. 
We also found from the documentation that at a V$_{GS}$ of 
5 V, the current I$_D$ was 1 A. We can find k$_n$ from this state
using the equation $I_D=\frac{1}{2}k_n(V_{GS}-V_{TH})^2$. The
value of k$_n$ that we found from that equation was k$_n=0.237812$.

With that value of k$_n$, we were able to use that same equation, but
with a current of I$_D=1$ mA to solve for a new V$_{GS}$ of 
2.1917 V. Knowing that V$_{GS}=2.1917$ V, we can find V$_S$.
Using the equation V$_S=$V$_G-$V$_{GS}$, we found that
V$_S=0-2.1917=-2.1917$.

To finally solve the circuit, we need to select the resistor
values that will give the specified values. R$_1$ and R$_2$ 
need to be the same value so that V$_G$ can equal 0, and 
with a higher resistance there will be less power used in
the circuit. The largest resistors available in the lab were
100~k$\Omega$ resistors, so we used those.

To find the value of R$_D$, we used Ohm's law. The voltage
over that resistor was $15-5=10$ V, and the current through
the resistor was 1 mA, so using $R=\frac{V}{i}$, we found
that the resistor we needed to use was 10~k$\Omega$. Likewise,
we used Ohm's law to find the value of the resistor R$_S$, the
difference in voltage was $-15-(-2.1917)=12.8083$ V. With just
1 mA of current, the resulting resistor value was
12.808~k$\Omega$.

To match those specific resistor values I wrote a python script
which takes a list of available resistors and shows a 
combination of resistors which can be combined in a resistor
network to get near that resistor value. We were able to get
within 1\% of the resistor value by combining a 15~k, 100~k, and
680~k resistors in parallel.


Based on this specification, V$_{OV}$ is V$_{GS}-$ V$_{TH}=2.1917-2.1=0.0917$ V.

\subsection{Simulation}

As shown in Figure \ref{fig:simulation1}, we simulated the circuit
and found that the current I$_D$ was just shy of 1~mA, and the
voltages V$_{G}$, V$_{D}$, and V$_S$ were very similar to our
calculated values.

\twopics{Part 1 Id}{Part 1 Vs Vd Vg}{fig:simulation1}{Results from the Simulation 1}

\subsection{Prototyping and Measurement}
After building the circuit we measured several values, shown in 
Figure \ref{fig:tablevalues1}.

\begin{figure}[!h]
\begin{center}
\begin{tabular}{ l|l }
  V$_G$ & -24.1 mV  \\
  V$_D$ & 5.32 V    \\
  V$_S$ & -2.30 V   \\
  V$_{GS}$ & 2.27 V \\
  V$_{DS}$ & 7.6 V  \\
  I$_D$ & 0.994 mA \\
  \multicolumn{2}{c}{Resistor Values} \\
  R$_S$ & 12.773~k$\Omega$ \\
  R$_D$ & 9.74~k$\Omega$   \\
  R$_1$ & 99.61~k$\Omega$  \\
  R$_2$ & 99.94~k$\Omega$
\end{tabular}
\caption{Table of Measured Values from Part 1}
\label{fig:tablevalues1}
\end{center}
\end{figure}

The measurements that we made were right in line with our
calculations, and we didn't get any interesting results
like we did in the lab last week.

\subsection{Post-Measurement Exercise}

Our measured values for V$_{GS}$ and V$_{DS}$ were very near
the calculated values. Small discrepancies could be explained by 
slightly different resistors or the threshold voltage of the 
transistor not being exactly 2.1 V.

We directly measured the value of I$_D$ in the circuit and 
got a value of 0.994 mA. If we use the V$_D$ of 5.32~V, and 
R$_D$ of 9.74~k$\Omega$, with a V$_{DD}$ of 15~V, we get a
current of $\frac{15-5.32}{9740}=1.004$ mA. That's close to 
our measured value, and very close to the calculated value as
well.

\section{Triode Region Analysis/Experiment}

For this second part of the lab we're redesigning the circuit
so that I$_D=10$~mA, V$_D=2$~V, and V$_{DS}=0.1$~V. Once again
we're using the same rail and drain voltages as the first part
of the lab.

\subsection{Hand Calculations}
We are using the same circuit from Figure \ref{fig:maincircuit}
for the calculation in this section as well. As opposed to the
equation used in part 1, the equation we use to express I$_D$ 
is $I_D=k_n(V_{OV}-\frac{1}{2}V_{DS})V_{DS}$.
The k$_n$ value remains the same as was calculated in the 
first section. Using this, we're able to find the value of
V$_{OV}$ from the I$_D$ expression. The V$_{OV}$ value that
we found was 0.4705~V. We can then find V$_{GS}=2.5705$ from 
the expression V$_{OV}=$V$_{GS}-$V$_{TH}$.

From the defined values of V$_D=2$~V and V$_{DS}=0.1$~V, we
can find that V$_S=1.9$~V. Using V$_S$ and V$_{GS}$ we find
that V$_G=4.4705$~V.

To finish designing the circuit we need to select resistor
values, using a voltage divider to get V$_G=4.4705$~V we got
R$_1=100$~k$\Omega$ and R$_2=184.9$~k$\Omega$. We were able
to combine a network of resistors using the same script 
as above to get near to those values. 

Using the same Ohm's law method as above we were able
to find R$_D=1.3$~k$\Omega$ and R$_S=1.69$~k$\Omega$.

\subsection{Simulation}

As shown in Figure \ref{fig:simulation2}, we simulated the circuit
and found that the current I$_D$ was just shy of 10~mA, and the
voltages V$_{G}$, V$_{D}$, and V$_S$ were very similar to our
calculated values.

\twopics{Part 2 Id}{Part 2 Vs Vd Vg}{fig:simulation2}{Results from the Simulation 2}

\subsection{Prototyping and Measurement}

We build the circuit and found these measurements, shown
in Figure \ref{fig:tablevalues2}.

\begin{figure}[!h]
\begin{center}
\begin{tabular}{ l|l }
  V$_G$ & 4.731 V  \\
  V$_D$ & 1.885 V    \\
  V$_S$ & 1.836 V     \\
  V$_{GS}$ & 2.907 V   \\
  V$_{DS}$ & 48.32 mV  \\
  I$_D$ & 10.20 mA \\
  \multicolumn{2}{c}{Resistor Values} \\
  R$_S$ & 1.653~k$\Omega$ \\
  R$_D$ & 1.287~k$\Omega$   \\
  R$_1$ & 99.95~k$\Omega$  \\
  R$_2$ & 0.193~M$\Omega$
\end{tabular}
\caption{Table of Measured Values from Part 1}
\label{fig:tablevalues2}
\end{center}
\end{figure}

Those values were mostly right in-line with the calculated
values.

\subsection{Post-Measurement Exercise}

Those values were aligned very well with the calculated values,
aside from V$_{DS}$ which was a little higher than it should've
been. Overall, though, the values were very similar to
our calculations for an NMOS transistor in the triode region.

\section{Conclusion}

Overall we were able to analyze the transistor in two different
settings, both in the saturation region and the triode region.
The most important part of the calculations was understanding the
implementation of the equation that express I$_D$ for the
different operation regions. Without appropriately using those
equations you can't fully analyze the circuit. I also thought
it was interesting being able to read more information from 
the data sheets than I have in the past. The more we do these
labs, the more I understand why transistors are so important
in the worlds of electrical and computer engineering.

\end{document}