\documentclass[12pt]{article}

%
%Margin - 1 inch on all sides
%
\usepackage{hyperref}
\usepackage[letterpaper]{geometry}
\usepackage{times}
\geometry{top=1.0in, bottom=1.0in, left=1.0in, right=1.0in}

%
%Doublespacing
%
\usepackage{setspace}
\singlespacing

%
%Rotating tables (e.g. sideways when too long)
%
\usepackage{rotating}


%
%Fancy-header package to modify header/page numbering (insert last name)
%
\usepackage{fancyhdr}
\pagestyle{fancy}
\lhead{} 
\chead{} 
\rhead{Davis \thepage} 
\lfoot{} 
\cfoot{} 
\rfoot{} 
\renewcommand{\headrulewidth}{0pt} 
\renewcommand{\footrulewidth}{0pt} 
\setlength{\headheight}{14.5pt}
%To make sure we actually have header 0.5in away from top edge
%12pt is one-sixth of an inch. Subtract this from 0.5in to get headsep value
\setlength\headsep{0.333in}

%
%Bibtex and Bibliography
%
\usepackage[american]{babel}
\usepackage{csquotes}
\usepackage[style=mla,backend=biber]{biblatex}
\addbibresource{refs.bib}


%
%Begin document
%
\begin{document}
\begin{flushleft}

%%%%First page name, class, etc
Ty Davis\\
Dr. Justin Jackson\\
ECE 3110\\
8 December 2023\\


%%%%Title
\begin{center}
  Alternative Semiconductor Materials
\end{center}



%%%%Changes paragraph indentation to 0.5in
\setlength{\parindent}{0.5in}
%%%%Begin body of paper here

The majority of the content of our ECE 3110 class
has considered the use of silicon in semiconductor
devices, but silicon is not the only semiconductor
material available. Through the course of this paper
I will consider the other types of semiconductor
materials that are used commonly, how and why they're
used, and their differences when compared with
silicon. Semiconductor devices have changed the way
our world works, and further developments are constantly
being researched in an attempt to find devices that
can do more for less energy used. The uses of 
semiconductor devices extend far beyond computation
devices such as computer processors. The robust 
utility of semiconductors means that they've found
their way into just about every electronic device 
we see today, and the magic behind a semiconductor
lies in the fact they possess properties of both a
conductor and an insulator. 

The discovery of these electrical properties of semiconductors
can be attributed to many people and began over 150
years ago.  The very first instances of discovering
semiconductor properties in materials can even date
back to Faraday's observations regarding electrical
conductivity in relation to temperature. This study
dates back to the year 1833, but it would be a long
time before people began thinking of semiconductors
as such extremely useful devices. Another important
discovery in the early development of semiconductor
thought includes the experiment of Alexandre-Edmond
Becquerel. When he experimented with the electrical
properties of electrolytes, he one day "noted that,
if one of the electrodes was illuminated with sunlight,
the emf generated between the electrodes increased."
\parencite{TudorJenkins_2005}. These first discoveries
of semiconductor properties opened the gateway to many
new ways to experiment with electricity. Doors were
opened and discoveries began to poor in. Other important
discoveries in the timeline of semiconductors include
observations from Willoughby Smith and Heinrich Hertz,
who had similar findings regarding photoconductivity
in the late 1800s, and another late 1800s discovery
regarding rectification in the contacts between metals
and some oxides and sulfides. Discoveries like these
poured out over the course of several decades and led
to increased thought and innovation in the area.

Moving forward to the 1920s and '30s, there was a significant
increase in demand for radar and other forms of communication
due to the impending war. Under this pressure, and
with the recent discovery of bipolar conduction in
semiconductors (being the flow of current due to both
electrons and holes moving), the scientific scene was
ready for heavy increase in the development of semiconductor
devices. The two primary material subjects of semiconductor
research were silicon and germanium, though many materials
possess the desired qualities in semiconductor behavior.
Discoveries surrounding the p-n junction and its manufacture
paved the way to new applications of diodes that could
support more current flow. This eventually led to the
discovery of transistors.






\newpage
\nocite{*}
\printbibliography

\end{flushleft}
\end{document}