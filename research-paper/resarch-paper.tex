\documentclass[12pt]{article}

%
%Margin - 1 inch on all sides
%
\usepackage{hyperref}
\usepackage[letterpaper]{geometry}
\usepackage{times}
\geometry{top=1.0in, bottom=1.0in, left=1.0in, right=1.0in}

%
%Doublespacing
%
\usepackage{setspace}
\onehalfspacing

%
%Rotating tables (e.g. sideways when too long)
%
\usepackage{rotating}


%
%Fancy-header package to modify header/page numbering (insert last name)
%
\usepackage{fancyhdr}
\pagestyle{fancy}
\lhead{} 
\chead{} 
\rhead{Davis \thepage} 
\lfoot{} 
\cfoot{} 
\rfoot{} 
\renewcommand{\headrulewidth}{0pt} 
\renewcommand{\footrulewidth}{0pt} 
\setlength{\headheight}{14.5pt}
%To make sure we actually have header 0.5in away from top edge
%12pt is one-sixth of an inch. Subtract this from 0.5in to get headsep value
\setlength\headsep{0.333in}

%
%Bibtex and Bibliography
%
\usepackage[american]{babel}
\usepackage{csquotes}
\usepackage[style=mla,backend=biber]{biblatex}
\addbibresource{refs.bib}


%
%Begin document
%
\begin{document}
\begin{flushleft}

%%%%First page name, class, etc
Ty Davis\\
Dr. Justin Jackson\\
ECE 3110\\
8 December 2023\\


%%%%Title
\begin{center}
  Alternative Semiconductor Materials
\end{center}



%%%%Changes paragraph indentation to 0.5in
\setlength{\parindent}{0.5in}
%%%%Begin body of paper here

The majority of the content of our ECE 3110 class
has considered the use of silicon in semiconductor
devices, but silicon is not the only semiconductor
material out there. Through the course of this paper
I will consider the other types of semiconductor
materials that are used commonly, where and why they're
used, and their differences when compared with
silicon. Semiconductor devices have changed the way
our world works, and further developments are constantly
being researched in an attempt to find devices that
can do more for less energy used. The uses of 
semiconductor devices extend far beyond computation
devices such as computer processors. The robust 
utility of semiconductors means that they've found
their way into just about every electronic device 
we see today. The magic behind a semiconductor lies
in the fact they possess properties of both a
conductor and an insulator. 











\newpage


\newpage
\nocite{*}
\printbibliography

\end{flushleft}
\end{document}