\documentclass[12pt]{article}

%
%Margin - 1 inch on all sides
%
\usepackage{hyperref}
\usepackage[letterpaper]{geometry}
\usepackage{times}
\geometry{top=1.0in, bottom=1.0in, left=1.0in, right=1.0in}

%
%Doublespacing
%
\usepackage{setspace}
\singlespacing

%
%Rotating tables (e.g. sideways when too long)
%
\usepackage{rotating}


%
%Fancy-header package to modify header/page numbering (insert last name)
%
\usepackage{fancyhdr}
\pagestyle{fancy}
\lhead{} 
\chead{} 
\rhead{Davis \thepage} 
\lfoot{} 
\cfoot{} 
\rfoot{} 
\renewcommand{\headrulewidth}{0pt} 
\renewcommand{\footrulewidth}{0pt} 
\setlength{\headheight}{14.5pt}
%To make sure we actually have header 0.5in away from top edge
%12pt is one-sixth of an inch. Subtract this from 0.5in to get headsep value
\setlength\headsep{0.333in}

%
%Bibtex and Bibliography
%
\usepackage[american]{babel}
\usepackage{csquotes}
\usepackage[style=mla,backend=biber]{biblatex}
\addbibresource{refs.bib}


%
%Begin document
%
\begin{document}
\begin{flushleft}

%%%%First page name, class, etc
Ty Davis\\
Dr. Justin Jackson\\
ECE 3110\\
8 December 2023\\


%%%%Title
\begin{center}
  Alternative Semiconductor Materials
\end{center}



%%%%Changes paragraph indentation to 0.5in
\setlength{\parindent}{0.5in}
%%%%Begin body of paper here

The majority of the content of our ECE 3110 class
has considered the use of silicon in semiconductor
devices, but silicon is not the only semiconductor
material available. Through the course of this paper
I will consider the other types of semiconductor
materials that are used commonly, how and why they're
used, and their differences when compared with
silicon. Semiconductor devices have changed the way
our world works, and further developments are constantly
being researched in an attempt to find devices that
can do more for less energy used. The uses of 
semiconductor devices extend far beyond computation
devices such as computer processors. The robust 
utility of semiconductors means that they've found
their way into just about every electronic device 
we see today, and the magic behind a semiconductor
lies in the fact they possess properties of both a
conductor and an insulator. 

The discovery of these electrical properties of semiconductors
can be attributed to many people and began over 150
years ago.  The very first instances of discovering
semiconductor properties in materials can even date
back to Faraday's observations regarding electrical
conductivity in relation to temperature. This study
dates back to the year 1833, but it would be a long
time before people began thinking of semiconductors
as such extremely useful devices. Another important
discovery in the early development of semiconductor
thought includes the experiment of Alexandre-Edmond
Becquerel. When he experimented with the electrical
properties of electrolytes, he one day "noted that,
if one of the electrodes was illuminated with sunlight,
the emf generated between the electrodes increased."
\parencite{TudorJenkins_2005}. These first discoveries
of semiconductor properties opened the gateway to many
new ways to experiment with electricity. Doors were
opened and discoveries began to poor in. Other important
discoveries in the timeline of semiconductors include
observations from Willoughby Smith and Heinrich Hertz,
who had similar findings regarding photoconductivity
in the late 1800s, and another late 1800s discovery
regarding rectification in the contacts between metals
and some oxides and sulfides. Discoveries like these
poured out over the course of several decades and led
to increased thought and innovation in the area.

Moving forward to the 1920s and '30s, there was a significant
increase in demand for radar and other forms of communication
due to the impending war. Under this pressure, and
with the recent discovery of bipolar conduction in
semiconductors (being the flow of current due to both
electrons and holes moving), the scientific scene was
ready for heavy increase in the development of semiconductor
devices. The two primary material subjects of semiconductor
research were silicon and germanium, though many materials
possess the desired qualities in semiconductor behavior.
Discoveries surrounding the p-n junction and its manufacture
paved the way to new applications of diodes that could
support more current flow. This eventually led to the
discovery of transistors.

The advent of transistors led the way to many new technologies,
and silicon quickly became the favorite material for
semiconductor development. The main reasons that silicon
is used in integrated circuits today are plenty. Silicon
is one of the most abundant resources found here on
Earth, meaning that it is relatively cheap and was
more readily available for experimentation and testing.
More important than cost, though, was the material
properties and electrical properties that made it so
special. Silicon boasts amazing reliability in diverse
conditions, and has shown the ability to operate as
intended in extreme weather and temperature conditions.
As silicon was adopted in the industry, manufacturers
would decide to make devices out of silicon so that
they could be compatible with already built devices,
and so that the manufacturing processes could source
from others as well.

The current methods for manufacturing silicon chips
have been through many renditions and revisions to get
where they are today. Each manufacturing company may have
their own processes to build the chips, but the process
roughly follows this same outline.  Silicon chips are
cut from wafers of silicon. After a purification process
to find produce a raw crystal of silicon, thin wafers
are sliced from a large ingot.  They are then cleansed
and polished to make sure that the surface is free
and defect free. Any small defects in the wafer now
can prevent operation in the integrated chip once it's
done. Once they've produced a clean silicon wafer it
undergoes a process that produces an oxidized SiO$_2$
layer on top of the silicon which acts as an insulating
material. The insulated silicon wafer then undergoes
a photolithography stage which defines the patterns
for the various components on the integrated circuit.
After an etching process, the silicon wafer undergoes
what is possibly the most important step when considering
the use of semiconductors. This is the doping stage, where
impurities are introduced into the silicon and modify its
electrical properties. Creating separate regions of p-type
and n-type concentrated areas allows the semiconductor
silicon to act as diodes and transistors. These produce
the computational logic of the chip and allow it to
work as intended. Often times, a silicon wafer undergoes
lithography, etching, doping and deposition multiple
times to correctly build the proper design on the 
silicon chip.

These manufacturing techniques have been developed
over years of experimentation with silicon chips, thus,
silicon quickly became king of the semiconductor field.
The discoveries, developments, and innovations in the
realm of electronics have increased greatly in this
era of silicon dominated semiconductor materials, but,
as we saw from years of electronic research and experimentation,
silicon isn't alone in its electrical properties. Other
materials are commonly used as semiconductors in electronics
today, including germanium, gallium arsenide, gallium
nitride, and others. Their strengths are separate from
those of silicon, thus they find their own place in
the world of semiconductor use.  Many semiconductors,
including the ones just previously listed, are made
of alloys consisted of Group III and Group V elements
on the periodic table. Gallium belongs to Group III
and has been paired in experimentation with several
Group V elements and found to have unique properties
that make it especially useful in specific areas, such
as the area of optoelectronics.

Gallium arsenide (GaAs) has been found very useful
in the area of optoelectronics. It has a higher electron
mobility than silicon, which makes it useful in high
frequency applications, such as radar systems, satellite
communication devices, etc. Research is currently being
done in the realm of building computer processors out
of GaAs because of its strengths in these areas which
could allow it to overcome some shortcomings of silicon-based
semiconductor devices. The major roadblocks that keep
silicon in the spotlight when it comes to semiconductor
devices, is the cost of materials and manufacturing.
One of the most common processes required for the manufacturing
of GaAs semiconductor devices is the use of molecular
beam epitaxy.

Molecular beam epitaxy (MBE) isn't just a process used
for the creation of GaAs semiconductor devices, it
can also be used for other Group III-V semiconductor
materials. The MBE process must be completed in an
ultra-high vacuum (UHV) environment, which is one of
the leading factors in how expensive the process is.
However, without a UHV environment, the process would
allow impurities into the semiconductor material, and
it would result in a poor outcome, so it is essential.
Similar to silicon wafers, after a crystal structure
is made from the semiconductor material, thin wafers
are cut from the crystal and prepared for processing.
Once the wafers of material are prepared, the effusion
cells are heated, evaporating the materials that will
be deposited on the surface of the wafer. This forms
molecular beams.





\newpage
\nocite{*}
\printbibliography

\end{flushleft}
\end{document}