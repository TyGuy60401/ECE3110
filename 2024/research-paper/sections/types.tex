% !TeX root = ..\research_paper.tex

The rectifier circuit roughly explained earlier includes
only a diode and a capacitor (acting as a reservoir
capacitor) to transfer an AC source into a DC output.
This rectifier circuit is commonly referred to as a
half-wave rectifier because it only allows half of the
input wave to come through. The benefits of a half-wave
rectifier are few, though there are certainly use cases
for it. The largest benefit of a half-wave rectifier
is the low cost. Because it uses only a single diode
and capacitor, the circuit is extremely cheap to manufacture.
On the other hand, the ripple factor of the rectifier
circuit is significantly higher when using a half-wave
rectifier circuit.

An alternative to the half-wave rectifier is the full-wave
rectifier. The full-wave rectifier behaves mathematically
as the absolute value of the input source. When the
AC input shows a negative voltage, it is effectively
inverted and becomes a equivalent positive voltage on
the output. These can be realized in several different
combinations, but the most common are the center-tap
and bridge rectifiers. A full bridge rectifier uses
a net of 4 diodes that are configured in such a way
that when the voltage from the AC source is positive
the current flows through half of the circuit, and when
the AC source is providing negative voltage the circuit
flows through the other half. The voltage flowing through
the negative half of the circuit is coming from the
negative terminal of the AC source, and because it is
in its negative cycle the voltage appears positive to
the relative ground, and the voltage on the output is
positive as well. This configuration allows the input
waveform to pass through when it is positive, and when
it is negative it is effectively inverted and allowed
to pass through once again. The downside of the full-bridge
rectifier circuit is that it takes four diodes to accomplish
this task. This makes it significantly more expensive
when compared with the half-wave rectifier from earlier,
but it is much easier to remove ripple because the output
waveform reaches the maximum twice as often.

An alternative to the bridge rectifier circuit is the
center-tap rectifier. A downside of this circuit is
that it can only be used in conjunction with a transformer
on the input. This is because the theory of a center-tap
rectifier relies on attaching the very center of the
transformer to the ground reference, so that when the
AC input switches from positive to negative, the negative
terminal of the center-tapped transformer appears positive
to that side of the circuit. Then, by placing a diode
on each end of the transformer and pointing towards
the load, you receive a full-wave rectified output.
While it requires the use of a transformer on the input,
this usually isn't an issue because a transformer is
likely already in place, especially if the circuit is
being built to use the input from something like the
wall power in your home which is fit for general use.
This rectifier provides a benefit over the full-bridge
rectifier circuit because it uses only two diodes instead
of four, which makes it more cost effective.

