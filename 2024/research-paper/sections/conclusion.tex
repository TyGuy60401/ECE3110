% !TeX root = ..\research_paper.tex

There is no doubt that rectifier research will persist
as time goes on. Historically, numerous different approaches
have been taken in order to solve the problem of converting
an AC source into a DC output. While the attempts to
rectify an AC source looked wildly different in the
early history of electrical discovery, the modern day
solutions seem to have been optimized. The vast majority of
rectifier circuits and devices in the modern day
are constructed with silicon or germanium based diodes,
and that doesn't seem like it will change any time soon.

Using those diodes in the purpose of rectifying small
power systems (those around 3-20~V such as consumer
electronics), is extremely common because they fit the
purpose so well. But, as the solutions have become more
and more similar, so have the problems. As consumer
electronics bend toward more consistent practices, the
designers choose to rely on existing technologies and
solutions to the problems that they face. Additionally,
the standardization of circuit design practices leads
to the increased consumption of similar componens. As
supply follows the demand, the cost of those components
will decrease and the designers of more consumer electronics
will choose to utilize those circuits and designs in
further products.

A wide range of rectifier circuits are used for a wide
range of purposes. I think that the developments in
high power WPT systems are interesting and could revolutionize
several different markets, including the electrical vehicle
market. Of course, rectifying an input is usually
only one component in a larger system, and there
might be other areas where optimization plays a larger
role than in the well-defined rectifier space. That being
said, advances in the field of electrical engineering
merit the expertise and perfection of every part of
a given electrical system.
