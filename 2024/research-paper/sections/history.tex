% !TeX root = ..\research_paper.tex

The first rectifier circuits that were invented relied
on devices that are no longer in use because of their
lack of reliability and relative safety concerns. An
example of such a device is the electrolytic rectifier,
which was made by placing two different metals in an
electrolytic solution such that direct current flowing
one way sees less resistance than in the other direction.
Such a rectifier was unsafe and unreliable because it
was temperature sensitive and some models would not
be able to operate in temperatures as low as 86 degrees
Fahrenheit. An electrolytic rectifier also suffered 
from a breakdown voltage when reverse biased which
limited the number of useful applications for the device.

Other early rectifiers/diodes included plasma type rectifiers
that took several different forms. A common rectifier
in use in the early 1900's was the mercury-arc rectifier.
Such a rectifier used a cathode submerged in liquid
mercury with an anode suspended above the pool. The
rectifer relied on an arc to form between the cathode
pool and the suspended anode, and could sustain high
power levels and up to six phases of AC current.

In the early development of radio technology, the need
for rectifying an AC radio signal gave birth to the
crystal detector. While these are no longer in use,
they served as some of the first semiconductor devices
and can date back to as early as the 1870's.
