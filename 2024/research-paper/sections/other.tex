% !TeX root = ..\research_paper.tex


Rectifiers convert AC sources into a DC output, but
they aren't only used for the sake of power delivery.
As I learned in my study of the crystal detector, rectifier
circuits are also used to demodulate an AM radio signal.
Because the signals are transmitted as AC signals with
the varying amplitude of the signal encoding the ``message'',
the AC signal needs to be demodulated in order to receive
the original waveform being transmitted.


A rather novel area of study is the concept of wireless
power transfer (WPT). Wireless power transfer relies
on the concept of radio-frequency (RF) to direct current
conversion, which is another way to think of rectification.
With WPT, it is essential to make sure that each part
of the pipeline is as efficient as possible, especially
when it is being employed to transfer a lot of power
quickly such as in the case of electrical vehicles.
The benefits of being able to use WPT for EV's are apparent,
but WPT can be required in some use cases, such as in
implantable devices where physical contact may not be
an option. When an RF signal is conducted wirelessly,
it is conducted as an alternating signal, and rectification
is needed to transform the output to a DC source suitable
for power delivery. There is a lot of research currently
being put toward maximizing the efficiency of rectifiers
because any loss in efficiency can have a large impact
in the performance of a system.


I found it interesting when an engineer needed to increase
the DC voltage in a system and decided the best way
to do that was by inverting the DC voltage to an AC
signal, then using a transformer to step up the voltage,
and finally rectifying the resulting AC waveform. It
didn't occur to me that this can be the best way to
step up a DC voltage, but that makes it even more important
to make sure that the rectification (and inversion)
processes are as efficient as possible.
