% !TeX root = ..\research_paper.tex

Alternating current, as shown by its name, alternates
between a positive current and a negative current as
time progresses, and as such it pushes and pulls the
current in whatever circuit is connected to it. Modern
rectifier circuits are built with semiconductor diodes
(usually made of silicon) and operate on the idea that
a diode only conducts eletricity in one direction. So,
when the AC source is pushing current through the diode,
the diode allows the current to flow through, but when
the AC source attempts to pull the current back, the
diode stops the flow of current, preventing a negative
voltage from appearing across the diode.

This phenomenon makes it so that the negative component
of the AC input is completely removed, and naturally,
the positive component is all that remains (\cite{sinclair1987electronics}).
The resulting average DC voltage that results is hence
positive, whereas the average voltage of the AC input
is always 0~V. The output of that rectifier is then
put through other stages of processing to change its
waveform from a half-sine shape to a steady DC voltage.

It is good to note that current flow through a diode
is not ideal as it has been explained up to this point.
The current through a diode does not occur when the
voltage is only just above 0~V. As we learned in lecture
this year, the voltage required across a diode such
that it allows current to pass through can be approximated
with the constant drop model at 0.7~V. The current-voltage
characteristic of a diode is not simple, but that rough
approximation is enough in most situations. This can
have an affect on the operation of a rectifier circuit.
When the AC input to a rectifier circuit is relatively
small, say around 5~V$_{pk}$, that 0.7~V drop can make
a significant difference in the DC output of the circuit.
On the other hand, if the required voltage output of
a rectifier is an order of magnitude or more higher,
then the 0.7~V drop won't have as much of an effect
and can be considered negligible.

As consumer electronics are being pushed towards lower
operating voltages, the 0.7~V drop is playing a bigger
role in the design of rectifier circuits. Other circuits,
such as precision rectifier circuits as will be covered
later in this paper, make use of op-amps or other components
to emulate an ideal diode and solve this problem.

When designing a rectifier circuit, there are some key
parameters of the output to keep in mind that can have
an effect on the performance of the device. Because
the translation of an AC source to a DC source is not
perfect, the output waveform will likely result in some
ripple. As such, the ripple factor is an important element
of consideration. If the device that requires a DC output
is sensitive to small changes in that DC power source,
a poorly designed rectifier circuit can cause it to
malfunction. As a reservoir capacitor is commonly used
to keep the voltage on the output from dipping with
the input, a heavy load on the circuit can cause a larger
ripple. ``When the current taken by the load is negligibly
small, the output voltage is almost perfectly smooth
DC. When an appreciable amount of current is being drawn,
however, the output contains fluctuations of lower amplitude.
This is because the reservoir capacitor is supplying
current during the time that the diodes are not conducting''
(\cite{sinclair1987electronics}).

Another important consideration when designing a rectifier
circuit is the peak inverse voltage (PIV). The PIV is
the voltage that the diode experiences when it is reverse
biased and not conducting electricity. If the PIV exceeds
the breakdown voltage of the diode in use, it can lead
to catastrophic effects as the voltage across the diode
will suddenly appear negative. A suggestion is made to be
conscious of the breakdown voltage of a specific diode when building
a circuit, ``It is usually prudent ... to select a diode
that has a reverse breakdown voltage at least 50\% greater than
the expected PIV'' (\cite{sedra2010microelectronic}).

