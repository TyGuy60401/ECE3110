% !TeX root = ..\research_paper.tex

This paper will be an exploration of the application
and types of rectifier circuits, and will include some
of their history and the significant people that contributed
to their discovery and use. A rectifier is a circuit
which is used to change an AC voltage into a DC voltage.
The problem of changing an AC voltage to a DC voltage
is not a recent one, and has been tackled many times
by many different people as early as the early 1900's.

Having access to DC power is important because the majority
of electronics operate on DC power. Despite DC power
being extremely common, the transmission of electric
power is more efficient when transported as AC, which
is a conversation that dates back to the times of Nikola
Tesla and Thomas Edison. Regardless, AC power is used
for transmitting electricity, and DC power is used by
many consumer electronic devices, so we need a way to
change between the two. A rectifier circuit (so called
because it straightens out an alternating current) is
the circuit used to produce a DC voltage when provided
with an AC voltage. As such, if you were to break open
a device that plugs into a wall socket and inspect the
circuits inside, you would very likely find a rectifier
circuit.

As noted, DC voltage is used by the majority of consumer
electronics, including cell phones, computers, televisions,
microwaves, and modern LED lights to name a few. It
is likely that in a given day you will interact with
devices that collectively rely on hundreds of rectifier
circuits, so efficient conversion of AC to DC is desirable
and can have a profound impact on both the energy consumption,
and the amount of components that are produced so that
any device can operate properly.
