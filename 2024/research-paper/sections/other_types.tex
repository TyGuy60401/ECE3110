% !TeX root = ..\research_paper.tex

There are also other forms of rectifiers which are not
discussed in the textbook. Such rectifiers rely on other
types of components, such as the thyristor. The thyristor
can be thought of as a two BJT transistors overlaid
such that they share a PN junction. The thyristor is
similar in theory to a voltage controlled diode, but
when the thyristor's gate is supplied with current the
thyristor doesn't turn off until the forward current
is removed. Similar to diodes, the thyristor does not
allow reversed current. The thyristor is also known
by the name \emph{silicon-controlled rectifier} (SCR).
It differs from a typical diode because it blocks forward
current if there isn't a gate pulse provided.

Because a gate pulse can be used to select the exact
time that the SCR starts conducting, using a SCR in
a rectifier circuit can allow for more control in the
output of the rectifier circuit that it is being used
in. Changing the time when the SCR starts conducting
relative to the period of the AC waveform allows for
control over the resulting DC voltage that the output
of the circuit sees. Despite being called a rectifier,
the SCR can be applied in the design of an inverter,
which converts a DC source into an AC output, essentially
the opposite of a rectifier.

Another benefit of the thyristor is the large amount
of forward current that it is able to deliver. In high-power
applications, the thyristor is an effective tool for
accurate rectification. Such devices that might rely on
high-power rectification are items such as ``dimmable lamps,
power regulators, and motors'' (\cite{thyristorapplications}). 

Rectifier circuits make an appearance in the design
of switched-mode power supplies (SMPS). Because an SMPS
switches the internal components on and off rapidly
to preserve power and improve efficiency, the device
will require either at least one rectifier on the output.
An SMPS will also usually be powered by some form of
AC source, so the input will need to be rectified as
well before being put through the pipeline of the power
supply. The rectifier on the output of the SMPS can
often be improved by the use of a Schottky diode. A
Schottky diode differs from a normal diode because it
is a semiconductor material placed next to a metal instead
of another semiconductor material. Schottky diodes possess
a number of advantages over their standard PN-junction
counterparts. They possess a lower forward voltage drop
--- usually around 0.2~V to 0.4~V --- than a standard
0.7~V diode, which allows them to be more efficient
on the output rectification step of the SMPS. This is
even more crucial for low-voltage SMPS designs, such
as those used in 3.3~V or 5~V devices.

Another advantage of the Schottky diodes in an SMPS
is the high switching speed that they possess. This
is because they lack the charge storage associated with
the P-N junction, and as such don't feel the same diffusion
capacitance effect that the standard diodes feel. Despite
these advantages, Schottky diodes aren't always used
in place of normal diodes for a few reasons. Primarily,
the reverse breakdown voltage is lower than in a normal
diode, and the cost of Schottky diodes is higher on
average than a standard P-N junction diode.

