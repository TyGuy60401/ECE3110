\documentclass[a4paper]{article}
\usepackage{geometry}
\usepackage{amsmath}
\usepackage{multicol}
\usepackage{graphicx}
\usepackage{siunitx}
\geometry{margin=0.25in}

\setlength{\parindent}{0pt}


\begin{document}
\begin{multicols}{3}
\raggedright

$f = 1/T$

$\omega = 2 \pi f$

Voltage divider:\\
$V_{r1} = v_{i} \frac{R1}{R1 + R2}$

\hrule

\textbf{Voltage and Current gain in decibels}: \\
$A_\textnormal{v-dB} = 20 \log_{10} ( A_\textnormal{V/V} )$ \\
$A_\textnormal{i-dB} = 20 \log_{10} ( A_\textnormal{A/A} )$

\textbf{Power gain in decibels}: \\
$A_\textnormal{p-dB} = 10 \log_{10} ( A_\textnormal{W/W} )$ \\

The power gain in decibels is the average of the voltage 
and amperage gain in decibels.

\hrule

\textbf{Amplifier Models}:\\
$v_o = A_{v_o} v_i \frac{R_L}{R_L + R_o}$ \\
$A_v = A_{v_o} \frac{R_L}{R_L + R_o}$

Consider the voltage divider at the output for the load.
There may also be one at the source getting into the
input of the amplifier.

\hrule

\textbf{Amplifier Types}:\\
\textbf{Voltage}: $A_{v_o} \equiv \frac{v_o}{v_i}$ \\
Wants: $R_i \rightarrow \infty$ and $R_o \rightarrow 0$

\textbf{Current}: $A_{i_s} \equiv \frac{i_o}{i_s}$ \\
Wants: $R_i \rightarrow 0$ and $R_o \rightarrow \infty$

\textbf{Transconductance}: $G_{m} \equiv \frac{i_o}{v_i}$ \\
Wants: $R_i \rightarrow \infty$ and $R_o \rightarrow \infty$

\textbf{Transresistance}: $R_{m} \equiv \frac{v_o}{i_s}$ \\
Wants: $R_i \rightarrow 0$ and $R_o \rightarrow 0$

\hrule

\textbf{Frequency Response of Amplifier}

STC (single time constant) networks:\\
$\tau =~$L / R  or  $\tau =~$CR

Low-pass filter with RC, voltage output is across the
capacitor because it appears short at high frequencies.

High-pass filter with RC, voltage output is across the resistor.

This is opposite for inductors.

\hrule

\textbf{Semiconductors}

Intrinsic means not doped. Group IV elements make good
semiconductors because they have 4 valence electrons.
At low temps, covalent bonds remain more intact and
don't conduct electricity. Higher temps means more free
electrons means more current.

\textbf{Recombination}
$n_i$ is number of free holes and electrons in a unit volume.

$n_i = BT^{3/2} e^{-E_g / 2kT}$

$B$ is material dependent parameter ($7.3 \times 10^{15} \si{\cm}^{-3}\si{\K}^{-3/2}$ for Si)

$E_g$ is band gap energy, in Si it's 1.12 eV.

$k$ is Boltzmann's Constant ($8.62 \times 10^{-5} \si{\eV}/\si{\K}$)

\hrule 

\textbf{Doped semiconductors}\\ 
\textbf{N-type}: Dope the semiconductor with a Group V element.
Those have an extra electron, so now there are extra
free electrons making it N-type.

If $N_D \gg n_i$ then $n_n \approx N_D$ \\
$n_n p_n = n_i^2$ \\
$p_n = \frac{n_i^2}{N_D}$

\textbf{P-type}: Dope the semiconductor with a Group III element,
more holes.


If $N_D \gg n_i$ then $n_n \approx N_D$ \\
$p_p n_p = n_i^2$ \\
$n_p = \frac{n_i^2}{N_A}$

\hrule

\textbf{Drift Current} \\
Hole velocity $v_{p-drift}=\mu_p E \frac{cm}{s}$\\
$\mu_p = 480 \frac{cm^2}{s}$

Electron velocity $v_{n-drift}=- \mu_n E \frac{cm}{s}$\\
$\mu_n = 1350 \frac{cm^2}{s}$

Hole current:
$I_p = Aqpv_{p-drift}$\\
$I_p = Aqp\mu_p E$\\
$J_p = \frac{I_p}{A} = qp\mu_p E$

$p$ is hole concentration

Electron current:
$I_n = -Aqnv_{n-drift}$\\
$I_n = -Aqn\mu_n E$\\
$J_n = \frac{I_n}{A} = qp\mu_n E$

$n$ is electron concentration
$q$ is magnitude of charge: $1.609 \times 10^{-19}$

\textbf{Drift current density}: \\
$J = J_p + J_n$

\hrule

\textbf{Diffusion Current}:
$J_p = -q D_p \frac{dp(x)}{dx} \frac{A}{cm^2}$\\
$J_n = q D_n \frac{dn(x)}{dx} \frac{A}{cm^2}$\\

$\frac{D_n}{\mu_n} = \frac{D_p}{\mu_p} = V_T$

$V_T$ is the thermal voltage.
$V_T = 25.9 \si{\mV}$

\hrule

\textbf{Junction Built-in Voltage}:\\
$V_0 = V_T \ln \Big( \frac{N_A N_D}{n_i^2} \Big)$

Usually, $V_0$ is between 0.6 and 0.9 V.

\textbf{Magnitude of Charge}:\\
$|Q_+| = qAx_n N_D$\\
$|Q_-| = qAx_p N_A$

$A$ is cross sectional area.

\hrule

\textbf{Width of depletion layer}:
$W = x_n + x_p = \sqrt{\frac{2 \varepsilon_s}{q} \Big(\frac{1}{N_A} + \frac{1}{N_D} \Big) V_0}$

$x_n = W \frac{N_A}{N_A + N_D} $ \\
$x_p = W \frac{N_D}{N_A + N_D} $

$Q_J = A \sqrt{ 2 \varepsilon_s q \Big( \frac{N_A N_D}{N_A + N_D} \Big) V_0}$

\textbf{With reverse voltage $V_R$}:
$Q_J = A \sqrt{ 2 \varepsilon_s q \Big( \frac{N_A N_D}{N_A + N_D} \Big) (V_0 + V_R)}$

Capacitance can be found by:

$C_j = \frac{dQ_J}{dV_R} \Big|_{V_R = V_Q}$\\
$C_j = \frac{\alpha}{2\sqrt{V_0 + V_R}}$ \\
Where $\alpha$ is shown by \\
$\alpha = A \sqrt{2 \varepsilon_s q \frac{N_A N_D}{N_A + N_D}}$


\hrule 

\textbf{Diffusion Capacitance}:\\
$Q_p = Aq [p_n(x_n) - p_{n0}] L_p$\\
$Q_p = \frac{L_p^2}{D_p} I_p$

$\tau_p = \frac{L_p^2}{D_p}$

$Q = \tau_T I$\\
$C_d = \frac{dQ}{dV}$

$C_d = \frac{\tau_T}{V_T} I$

Where $I$ is the forward bias current, it is really
small when the diode is reverse biased.

\hrule

\textbf{Depletion Capacitance}:\\

$C_{j0} = A \sqrt{\Big( \frac{\varepsilon_s q}{2} \Big) \Big( \frac{N_A N_D}{N_A + N_D} \Big) \frac{1}{V_0}} $
$C_j = \frac{C_{j0}}{\Big( 1 + \frac{V_R}{V_0} \Big) ^ m}$

$m$ is between 1/3 and 1/2


\hrule

\textbf{Saturation Current}:\\

$I_s = Aq n_i^2 \Big( \frac{D_p}{L_p N_D} + \frac{D_n}{L_n N_A} \Big)$

$I = I_s \big( e^{V/V_T} - 1\big)$

\textbf{Forward Current}:\\

$I = I_p + I_n$

$I_p = A q n_i^2 \frac{D_p}{L_p N_D} (e ^ {V/V_T} - 1)$

$I_n = A q n_i^2 \frac{D_n}{L_n N_A} (e ^ {V/V_T} - 1)$

\textbf{Diodes}:\\

$V_T = \frac{kT}{q} = 25.9 \si{\mV}$

\textbf{Iterative Solution}: \\
Guess $V_1$ and $I_1$

$I_{2} = \frac{V_{DD} - V_1}{R}$

$V_2 = V_1 + V_T \ln \Big( \frac{I_2}{I_1} \Big)$

Now use $V_2$ and $I_2$ as the new inputs for another iteration.

Constant drop: $V_D = 0.7~\si{\V}$.

\textbf{Small Signal}:

$r_d = \frac{V_T}{I_D}$

\textbf{Zener Regions}:

$V_Z = V_{Z0} + r_z I_Z$

\textbf{Rectifiers}:

Ripple voltage:

$V_r = \frac{V_p}{fCR} = \frac{I_L}{fC}$




\end{multicols}
\end{document}
